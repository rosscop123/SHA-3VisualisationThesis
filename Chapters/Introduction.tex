\chapter{Introduction}
\section{Motivation}
Since the start of the computer era, cryptography has become an ever more important piece of the world we live in. Although unknown to many, cryptography is performed throughout tasks we now take for granted. Gaining access to your 'electronic cash', e-messaging or simply viewing a web-page, all use some form of cryptography.
\vspace{5 mm}\\
Many hope for a world where security and privacy is no longer a worry, but everybody knows this is far from a reality and that will almost certainly remain the case. Cryptography is therefore a field which needs to continue to advance forward. This requires careful analysis of future and present standards. It may seem contradictory, but one of the best ways to do this is to attempt to break, or find weaknesses in, existing standards. This requires new tools to help cryptographers analyse current and upcoming standards.
\vspace{5 mm}\\
Not only is cryptography a difficult subject to study, it is evolving at an increasing rate, leading to problems when educating people about specific standards in cryptography. When new standards are released, the take up rate for them is extremely slow, predominantly because of a lack of trust towards the standard. Trust for the standard comes over time and the more programmers that understand the standard, the more likely they will be to include standards in small applications and thus begins to build trust of a standard within the computer science community.
\vspace{5 mm}\\
SHA-3 was a standard introduced in early 2015, meaning very few people understand how SHA-3 works and even fewer trust the algorithm, which is yet to be tested in real world applications. This gives rise to one of the main reasons to develop tools around such an algorithm; to ensure that computer scientists remain educated and up to date in fields which defends sensitive data and allowing them to analysis the algorithm for themselves.
\section{Statement Of Problem}
\label{sec:StatementProb}
Currently, SHA-3 has few tools available to computer scientists which aid in the teaching of the algorithm. This would be very beneficial to lecturers and students, significantly decreasing the time needed to explain the algorithm and then helping students during self-study of the algorithm. Not only is there a lack of tools in teaching the algorithm but also for analysis of SHA-3 and these tools will become ever more necessary when SHA-3 begins to gain momentum.
\section{General Aims}
The general aim of this project is to develop an application aiding in the teaching and analysis of SHA-3. The application will be able to calculate SHA-3 and breakdown each of the stages in the algorithm for the user, whilst also including tools which will allow for a better analysis of the algorithm. Above all, the tools provided must take a different approach to pre-existing software, which will be done by visualising the algorithm (explaining how we will visualise SHA-3 will be covered further into this thesis).
\vspace{5 mm}\\
Primarily this application will be aimed towards people looking to learn how SHA-3 works. Alongside this group of people, cryptographers looking to analyse the algorithm will benefit greatly from the application.
\vspace{5 mm}\\
The application must also be easily accessible for all who wish to use it, meaning not only does the application have to be cross platform but it also must run efficiently as possible, for users who require to run the application on older machines.
\section{Outline Of Thesis}
This thesis will be structured in a logical order such that the reader may follow the project through each of its stages and obtain a good understanding as to how each choice was made and how the development of the application occurred. Each remaining chapter contains the following information:
\vspace{5 mm}\\
Chapter 2 : This chapter will discuss the background of the project and the research that took place. It will look into other applications which may already exist with the same objectives, it will argue the reasoning behind the decisions and it will examine other research.
\vspace{5 mm}\\
Chapter 3 : This chapter will explain any decisions taken prior to developing the application, before going on to discuss the specification and the design of the application.
\vspace{5 mm}\\
Chapter 4 :
\vspace{5 mm}\\
Chapter 5 :